\documentclass{article}
\usepackage[utf8]{inputenc}

\usepackage{amsfonts}
\usepackage{amssymb}
\usepackage{amsmath}
\usepackage{amsthm}
\usepackage{enumitem}

\usepackage{bbold}
\usepackage{bm}
\usepackage{graphicx}
\usepackage{color}
\usepackage{hyperref}
\usepackage[margin=2.5cm]{geometry}

\begin{document}


% ==============================================================================

\title{\Large{INFO8006: Project 2 - Report}}
\vspace{1cm}
\author{\small{\bf Romain LAMBERMONT - s190931} \\ \small{\bf Arthur LOUIS - s191230}}

\maketitle

% ==============================================================================

\section{Problem statement}

\begin{enumerate}[label=\alph*.,leftmargin=*]
    \item In this section, we'll discuss the problem as an adverserial search problem and we'll define all elements of the problem
    \begin{itemize}
        \item \underline{Set of states :} A state in Pacman can be described as :
        \begin{equation*}
            s = \{\text{pacmanPos}, \text{ghostPos}, \text{ghostDirection}, \text{foodMatrix}, \text{score}\}
        \end{equation*}
        Where all terms are respectively : the position of Pacman, the position of the ghost, the direction of the ghost, the food matrix (representing food position with boolean values) and the current score.
        \\
        In this case we can define the initial state $s_0$ like this :
        \begin{equation}
            s_0 = \{\text{pacmanPos}_0, \text{ghostPos}_0, \text{ghostDirection}_0, \text{foodMatrix}_0, 0\}
        \end{equation}
        Where all terms with a $0$ index being initial paramaters given by the layout.
        \item \underline{Player function :} The player can easely be determined this way :
        \begin{equation*}
            \text{player}(s) =  \begin{cases}
                                    0 \text{ if Pacman}\\
                                    1 \text{ if Ghost}
                                \end{cases}
        \end{equation*}                       
        \item \underline{Legal actions :} In a specific state $s$, the set pf legal actions is :
        \begin{equation*}
            \text{action}(s) = \{\text{North, East, South, West}\}
        \end{equation*}
        These actions are only available under the condition that we don't hit a wall if we take the action, thus it depends of the layout.
        \item \underline{Transition model :} The transition model can be described as the result of the action taken on the previous state, giving us the new state $s'$:
        \begin{equation*}
            s' = \text{result}(s,a) = \{\text{pacmanPos} + a, \text{ghostPos'}, \text{ghostDirection'}, \text{foodMatrix'}, \text{score'}\}
        \end{equation*}
        Where all parameters with ' are computed as a response from the action taken $a$ (the ghost reacts at the action taken by Pacman and by taking consideration of its behaviour and the food and score are also updated as a result from the action $a$).
        \item \underline{Terminal test :} The test to check if the state is terminal is simple :
        \begin{equation*}
            \text{terminal}(s) = \begin{cases}
                                    \text{1 if ghost eats Pacman OR Pacman has eaten all dots in foodMatrix}\\
                                    \text{0 Otherwise}
                                 \end{cases}
        \end{equation*}
        \item \underline{Utility function} : The utility function of a Pacman is simply given by the score :
        \begin{equation*} \text{utility}(s,p) = \begin{cases}
                                                    \text{score if }p\text{ = Pacman}\\
                                                    -\text{score if }p\text{ = Ghost}
                                                \end{cases}
        \end{equation*}
        As a reminder the score value is computed using this formula :
        \begin{equation*}
            \text{score} = -\text{time steps} + 10 * \text{eaten dots} - 5 * \text{eaten capsules} + 200 * \text{eaten ghost} + (-500 \text{ if lost}) + (500 \text{ if won})
        \end{equation*}
    \end{itemize}
    \newpage
    \item In the case of a zero-sum game as Pacman the utility function can be described as :
    \begin{equation*}
        \text{utility}(s,p = \text{Ghost}) = \begin{cases}
                                \text{1 if Ghost eats Pacman}\\
                                \text{-1 if Pacman eats all dots}
                              \end{cases}
    \end{equation*}
\end{enumerate}

\section{Implementation}

\begin{enumerate}[label=\alph*.,leftmargin=*]
    \item For an algorithm to be complete, it has to find a solution if one exists. In minimax, we run through all the nodes to find a solution. This only applies if the game tree is finite, so we have to avoid repetition of moves resulting in looping on the same states.
    \\\\
    To avoid this, we keep track of the previously visited nodes and the value of the minimax computation respectively in a set and in a dictionnary (we check if the node is visited thanks to an key function). And by simply looking at the score function we can see 
    that going back on the same states is not useful. Indeed if we find the same node lower in the recursion tree, the score decreases at each time step, so we need to avoid already computed nodes.
    \\\\
    In conclusion, to make the algorithm complete, we would need to modify the set of Legal Moves, including a move to avoid already visited nodes.
    \item \textbf{\textit{Refer to minimax.py .}}
    \item \textbf{\textit{Refer to hminimax(0/1/2).py .}}
    \\\underline{Evaluation / cut-off functions :}
    \begin{itemize}
        \item Cut-off : In our 3 different implementations, we'll use the same cut-off function. Indeed, we found it quite simple but still really good with a computation time kept low. 
        Our function just checks if the game is finished (lost or won) or if the depth in the recursion tree exceeds 4 (best relation between computation time and score)
        \item HMinimax0 : 
        \item HMinimax1 :
        \item HMinimax2 : 
    \end{itemize}
\end{enumerate}

\section{Experiment}

\begin{enumerate}[label=\alph*.,leftmargin=*]
    \item
    \item
\end{enumerate}



% ==============================================================================

\end{document}